% pliki w unicode, polskie zasady typograficzne
\usepackage{polski}
\usepackage[utf8]{inputenc}

% wymiary wg. wytycznych pollub
\usepackage[left=35mm,right=20mm,top=25mm,bottom=20mm]{geometry}

% klikalne linki w PDF
\usepackage{hyperref}
% warto odkomentować oraz uzupełnić poniższe pola dokumentu PDF,
% jeżeli planowana jest publikacja elektroniczna pracy
\hypersetup{
%  bookmarks=true,
   pdfborder={0 0 0},
%  pdftitle={Temat Pracy},
%  pdfauthor={Autorj},
%  pdfkeywords={Słowa, Kluczowe}
}

% Naprawia problem z labelami obrazków, które muszą być wstawiane pod caption. Przez to link do obrazka przenosi do miejsca pod obrazkiem. Poniższe ustawienie powoduje przeniesienie do górnej krawędzi obrazka.
\usepackage[hypcap]{caption}

\usepackage{listings}
\renewcommand{\lstlistlistingname}{Spis listingów} % spolszczenie
%\usepackage{multicol}
\usepackage{graphicx}
\DeclareGraphicsExtensions{.eps}
\graphicspath{{./gfx/}{./dot/}}
\usepackage{epstopdf}

\usepackage{setspace}
\onehalfspacing

% zerowanie licznika przy 'figure','table','equation' w kazdym rozdziale
\usepackage{amsmath}
\numberwithin{figure}{section}
\numberwithin{table}{section}
\numberwithin{equation}{section}
% licznik lstlisting z uwagi na specyfikę implementacji został zdefiniowany za  \begin{document}

% zerowanie licznika 'footnote' na każdej stronie
\usepackage[perpage]{zref}
\zmakeperpage{footnote}

% alternatywne czcionki (preferowane są kroje szeryfowe)
%\usepackage{concmath} % http://www.tug.dk/FontCatalogue/ccr/
%\usepackage{ccfonts,eulervm}
%\usepackage[charter]{mathdesign}
%\usepackage{fourier}
%\usepackage{qtimes} % Times

% nagłówki -- fancy header
\usepackage{fancyhdr}
\pagestyle{fancy}
\setlength{\headheight}{12pt}
\renewcommand{\headrulewidth}{0.2pt}
\setlength{\headsep}{6pt}

% reset
\fancyhf{}

% stopka
\fancyfoot[RO, LE] {\thepage}						% dwustronny
%\fancyfoot[C] {\thepage}							% jednostronny

% nagłówek
\fancyhead[LE,RO]{\scriptsize\slshape \rightmark}	% dwustronny
\fancyhead[LO,RE]{\scriptsize\scshape \leftmark}	% dwustronny
%\fancyhead[R]{\scriptsize\slshape \rightmark}		% jednostronny
%\fancyhead[L]{\scriptsize\scshape \leftmark}		% jednostronny

\renewcommand{\sectionmark}[1]{\markboth{#1}{}} % brak numeru przy nazwie rozdziału w nagłówku
\renewcommand{\subsectionmark}[1]{\markright{\thesubsection. #1}{}} % fix brakującej kropki po numerze podrozdziału w nagłówku

%  plain page
\fancypagestyle{plain}{
\fancyhf{}
\fancyfoot[RO, LE] {\thepage}						% dwustronny
%\fancyfoot[C] {\thepage}							% jednostronny
\renewcommand{\headrulewidth}{0pt}
\renewcommand{\footrulewidth}{0pt}}

% globalne ustawienia listingów
\usepackage{textcomp} % wymagane przez upquote poniżej
\lstset{ %
  numbers=left,
  %xleftmargin=1.5em,
  stepnumber=1,
  numberstyle=\scriptsize,
  basicstyle=\ttfamily\small,
  %basicstyle=\small,
  %stringstyle=\ttfamily,
  keywordstyle=\bfseries\underbar,
  captionpos=t,
  breaklines=true,
  frame=leftline,
  breaklines=true,
  upquote=true,
  extendedchars=true, % na problemy ze znakami unicode
  literate =
    {ą}{{\k{a}}}1
    {Ą}{{\k{A}}}1
    {ę}{{\k{e}}}1
    {Ę}{{\k{E}}}1
    {ó}{{\'o}}1
    {Ó}{{\'O}}1
    {ś}{{\'s}}1
    {Ś}{{\'S}}1
    {ł}{{\l{}}}1
    {Ł}{{\L{}}}1
    {ż}{{\.z}}1
    {Ż}{{\.Z}}1
    {ź}{{\'z}}1
    {Ź}{{\'Z}}1
    {ć}{{\'c}}1
    {Ć}{{\'C}}1
    {ń}{{\'n}}1
    {Ń}{{\'N}}1
}

% Naprawia problem z niedziałającą opcją lstset{breaklines = true}
% Klasa dokumentu mwart zabrania łamania wierszy na dywizie i
% w jakiś sposób uniemożliwia to również działanie łamania wierszy
% wewnątrz lstlisting
\usepackage{etoolbox}
\AtBeginEnvironment{lstlisting}{\exhyphenpenalty = 0}
\AfterEndEnvironment{lstlisting}{\exhyphenpenalty = 10000} % taka wartość jest ustawiona w klasie mwart

% generator używany przez praca/demo.tex
\usepackage{lipsum}

% prosty skrot tworzacy paragrafy z~numerowanymi przykladami \przyklad -> Przyklad 4.2:
\newcommand{\przyklad}{
  \refstepcounter{przyklad}
  \paragraph{Przykład~\theprzyklad:}
}
\newcounter{przyklad}
\numberwithin{przyklad}{section} % numerowanie wg. wytycznych

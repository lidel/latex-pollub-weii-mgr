% pliki w unicode, polskie zasady typograficzne
\usepackage{polski}
\usepackage[utf8]{inputenc}

% wymiary wg. wytycznych pollub
\usepackage[left=35mm,right=20mm,top=25mm,bottom=20mm,nohead]{geometry}

% inne zależności
\usepackage[bookmarks=false,pdfborder={0 0 0}]{hyperref}
\usepackage{listings}
\renewcommand{\lstlistlistingname}{Spis listingów} % spolszczenie
%\usepackage{multicol}
\usepackage{graphicx}
\graphicspath{{./gfx/}{./dot/}}

\usepackage{setspace}
\onehalfspacing

% zerowanie licznika przy 'figure','table','equation' w kazdym rozdziale
\usepackage{amsmath}
\numberwithin{figure}{section}
\numberwithin{table}{section}
\numberwithin{equation}{section}
% licznik lstlisting z uwagi na specyfikę implementacji został zdefiniowany za  \begin{document}

% zerowanie licznika 'footnote' na każdej stronie
\usepackage[perpage]{zref}
\zmakeperpage{footnote}

% dodatkowe 'ciasne' wypunktowania
\newenvironment{packed_itemize}{
\begin{itemize}
  \setlength{\itemsep}{1pt}
  \setlength{\parskip}{0pt}
  \setlength{\parsep}{0pt}
}{\end{itemize}}
\newenvironment{packed_enumerate}{
\begin{enumerate}
  \setlength{\itemsep}{1pt}
  \setlength{\parskip}{0pt}
  \setlength{\parsep}{0pt}
}{\end{enumerate}}

% globalne ustawienia listingów
\usepackage{textcomp} % wymagane przez upquote poniżej
\lstset{ %
  numbers=left,
  %xleftmargin=1.5em,
  stepnumber=1,
  numberstyle=\scriptsize,
  basicstyle=\ttfamily\small,
  %basicstyle=\small,
  %stringstyle=\ttfamily,
  keywordstyle=\bfseries\underbar,
  captionpos=t,
  breaklines=true,
  frame=leftline,
  %showspaces=true,
  breaklines=true,
  upquote=true,
  %resetmargins=true;
  %float=[p!],
  %boxpos=t,
  %frameround=fftt,
}

% generator używany przez praca/demo.tex
\usepackage{lipsum}
